\section{System Design}


\subsection{Use Case Diagram}
The presented diagram covers the entirety of actions that the different kinds of entities can perform in EduMeter. The specific interactions with the system are covered extensively in the use case templates in section \ref{use-case-templates}.

\vspace{1em}
	
	\begin{tikzpicture}
		
		\umlusecase[name = browse-review ,x = 6    ,y = 0     ,width = 2.5cm] {Browse Review}
		\umlusecase[name = apply-filter  ,x = 12   ,y = -1.5  ,width = 2.5cm] {Apply Filter}
		\umlusecase[name = login         ,x = 9    ,y = -3    ,width = 2.5cm] {Login}
		\umlusecase[name = submit-review ,x = 14   ,y = -6.7  ,width = 2.5cm] {Submit Review}
		\umlusecase[name = up-vote       ,x = 6    ,y = -5    ,width = 2.5cm] {Up-Vote}
		\umlusecase[name = report-user   ,x = 9    ,y = -6.4  ,width = 2.5cm] {Report User}
		\umlusecase[name = login-admin   ,x = 4.5  ,y = -9    ,width = 2.5cm] {Login}				
		\umlusecase[name = manage-review ,x = 6.5  ,y = -11   ,width = 2.5cm] {Manage Reviews}				
		\umlusecase[name = delete-review ,x = 13   ,y = -12.5 ,width = 2.5cm] {Delete Review}
		\umlusecase[name = validate-data ,x = 13   ,y = -9.5  ,width = 3.0cm] {Validate Review Information}
		\umlusecase[name = manage-report ,x = 4.5  ,y = -13   ,width = 2.5cm] {Manage Report}
		\umlusecase[name = validate-rep  ,x = 10   ,y = -15   ,width = 2.5cm] {Validate Report}
		\umlusecase[name = ban-user      ,x = 5    ,y = -15.5 ,width = 2.5cm] {Ban User}
		
		\umlactor[y = -1]  {Guest}
		\umlactor[y = -5]  {Student}
		\umlactor[y = -11] {Admin}
		
		\umlassoc{Guest}   {browse-review}
		\draw [tikzuml association style](apply-filter)  edge[bend left  =  5]  (Guest);
   		
	    \umlassoc{Student} {up-vote}
		\umlinherit{Student} {Guest}
		\draw [tikzuml association style](report-user)   edge[bend left  = 10]  (Student);
		\draw [tikzuml association style](submit-review) edge[bend left  = 20]  (Student);
		\draw [tikzuml association style](login)         edge[bend right = 10]  (Student);
		
		\umlassoc{Admin}  {login-admin}
		\umlassoc{Admin}  {manage-review}
		\umlassoc{Admin}  {manage-report}
		
		\draw [tikzuml dependency style] (browse-review) edge[bend left  = 10] node[right] {$\ll \text{extends} \gg$} (apply-filter);
		\draw [tikzuml dependency style] (submit-review) edge[bend right = 35] node[right] {$\ll \text{include} \gg$} (login);
		\draw [tikzuml dependency style] (up-vote)       --                    node[left]  {$\ll \text{include} \gg$} (login);
		\draw [tikzuml dependency style] (manage-review) --                    node[above] {$\ll \text{include} \gg$} (validate-data);
		\draw [tikzuml dependency style] (delete-review) --                    node[right] {$\ll \text{extends} \gg$} (manage-review);
		\draw [tikzuml dependency style] (report-user)   --                    node[right] {$\ll \text{include} \gg$} (login);
		\draw [tikzuml dependency style] (manage-report) --					   node[right] {$\ll \text{include} \gg$} (validate-rep);
		\draw [tikzuml dependency style] (ban-user)      --					   node[left]  {$\ll \text{extends} \gg$} (manage-report);
	\end{tikzpicture}
	

\subsection{Use Case Template} \label{use-case-templates}



\subsubsection{Guest Templates}
\begin{usecasetable}{1}{Browse Review}
	\begin{tabularx}{\textwidth}{l|X}
		\textbf{Level} & User Goal \\
		\textbf{Description} & An unauthenticated user browses the collection of reviews to gather information. \\
		\textbf{Actors} & Guest (Student) \\
		\textbf{Pre-conditions} & None \\
		\textbf{Steps} & 
		\begin{tabenum}
			\item The guest accesses the EduMeter platform (Mock-up \ref{mock-up-1-landing-page}).
			\item The system displays a list of reviews.
			\item The guest scrolls through and reads them (Mock-up \ref{mock-up-2-browsing-page}).
		\end{tabenum} \\
		\textbf{Post-conditions} & The user has viewed the desired information. \\
		\textbf{Alternative Steps} & 
		\begin{tabitem}
			\item[2a.] \textbf{Extension (Apply Filter):} The guest wants to narrow down the search results:
			\begin{itemize}
				\item The guest selects specific criteria.
				\item The system updates the view to show only matching reviews.
			\end{itemize}
		\end{tabitem}
	\end{tabularx}
\end{usecasetable}


\subsubsection{Student Templates}

\begin{usecasetable}{2}{Student Login}
	\begin{tabularx}{\textwidth}{l|X}
		\textbf{Level} & Function \\
		\textbf{Description} & The user is authenticated via their institutional email. \\
		\textbf{Actors} & Student \\
		\textbf{Pre-conditions} & The user possesses a valid institutional email address. \\
		\textbf{Steps} &
		\begin{tabenum}
			\item The user provides their institutional email address (Mock-up \ref{mock-up-3-auth-page-step-1}).
			\item The system sends a verification code to the provided address.
			\item The user enters the code (Mock-up \ref{mock-up-4-auth-page-step-2}).
			\item The system verifies if the entered code matches the one sent.
			\item The system retrieves the existing unique ID (or generates a new one for first-time access).
		\end{tabenum} \\
		\textbf{Post-conditions} & The student is authenticated and identified by their unique ID. \\
		\textbf{Alternative Steps} & \begin{tabitem}
			\item[4a.] If the code does not match, the system signals an error and does not grant access.
		\end{tabitem}
	\end{tabularx}
\end{usecasetable}


\begin{usecasetable}{3}{Submit Review}
	\begin{tabularx}{\textwidth}{l|X}
		\textbf{Level} & User Goal \\
		\textbf{Description} & An authenticated user submits a review for a completed course. \\
		\textbf{Actors} & Student \\
		\textbf{Pre-conditions} & The user must be authenticated. \\
		\textbf{Steps} &
		\begin{tabenum}
			\item The user selects the option to create a review.
			\item The system displays the review form (Mock-up \ref{mock-up-7-review-form}).
			\item The user fills the review fields (school, degree, course, professor) by selecting existing option from a provided list.
			\item The user add a rating and a comment.
			\item The user confirms the submission.
			\item The system stores the review and makes it publicly visible.
		\end{tabenum} \\
		\textbf{Post-conditions} & The review is stored in the database. \\
		\textbf{Alternative Steps} & \begin{tabitem}
			\item[3a.] If the information that the student need is not available in the list: 
			\begin{itemize}
				\item The user manually inputs the missing information in the fields.
				\item Upon submission, the system saves the review but is not published publicly until an Admin approves it (UC \#5: Validate Review Data).
			\end{itemize} 
		\end{tabitem}
	\end{tabularx}
\end{usecasetable}


\begin{usecasetable}{4}{Up-Vote}
	\begin{tabularx}{\textwidth}{l|X}
		\textbf{Level} & User Goal \\
		\textbf{Description} & An authenticated Student expresses approval for a specific review. \\
		\textbf{Actors} & Student \\
		\textbf{Pre-conditions} & The student must be authenticated. \\
		\textbf{Steps} &
		\begin{tabenum}
			\item The student selects the upvote option on a specific review.
			\item The system checks if the student has already voted for this review.
			\item The system increments the vote counter for the review.
		\end{tabenum} \\
		\textbf{Post-conditions} & The vote is stored in the database, updating the review's total score. \\
		\textbf{Alternative Steps} & \begin{tabitem}
			\item[3a.] If the student has already voted, the system does not register the new vote.
		\end{tabitem}
	\end{tabularx}
\end{usecasetable}

\vspace{2em}
\begin{usecasetable}{5}{Report Review}
	\begin{tabularx}{\textwidth}{l|X}
		\textbf{Level} & User Goal \\
		\textbf{Description} & An authenticated student reports inappropriate content. \\
		\textbf{Actors} & Student \\
		\textbf{Pre-conditions} & The student must be authenticated. \\
		\textbf{Steps} &
		\begin{tabenum}
			\item The student reads a review with inappropriate content.
			\item The student selects the option to report the review.
			\item The student writes a brief comment for the report. 
			\item The system registers the report.
		\end{tabenum} \\
		\textbf{Post-conditions} & The report is visible to the admin for validation. \\
		\textbf{Alternative Steps} & None
	\end{tabularx}
\end{usecasetable}


\subsubsection{Admin Templates}

\begin{usecasetable}{6}{Admin Login}
	\renewcommand{\arraystretch}{1.4}
	\begin{tabularx}{\textwidth}{l|X}
		\textbf{Level} & Function \\
		\textbf{Description} & The admin log in the system to access his functionality. \\
		\textbf{Actors} & Admin \\
		\textbf{Pre-conditions} & The admin email must be stored in the database. \\
		\textbf{Steps} &
		\begin{tabenum}
			\item The admin accesses the dedicated administrative login interface.
			\item The admin provides their registered email address.
			\item The system verifies if the email is in the admin whitelist.
			\item The system sends a verification code to the provided address.
			\item The admin enters the code.
			\item The system verifies if the entered code matches the one sent.
			\item The system notifies the successful login.
		\end{tabenum} \\
		\textbf{Post-conditions} & The admin is logged in and has access to his functionality. \\
		\textbf{Alternative Steps} & \begin{tabitem}
			\item[3a.] If the provided email is not recognized, the system signal the error and ask the admin to provide a new one.
			\item[7a.] If the provided code don't match the one send, the system ask the Admin to re-enter the code or request a new one.
		\end{tabitem}
	\end{tabularx}
\end{usecasetable}


\begin{usecasetable}{7}{Validate Review Information}
	\begin{tabularx}{\textwidth}{l|X}
		\textbf{Level} & User Goal \\
		\textbf{Description} & The admin verifies the submitted information. \\
		\textbf{Actors} & Admin \\
		\textbf{Pre-conditions} & The admin is logged in. \\
		\textbf{Steps} &
		\begin{tabenum}
			\item The admin selects a review from the queue of pending validations (Mock-up \ref{mock-up-5-admin-control-panel}).
			\item The admin links the following fields to existing database entries (Mock-up \ref{mock-up-6-admin-edit-panel}):
			\begin{itemize}[leftmargin=1.5em, nosep]
				\item School
				\item Degree
				\item Course
				\item Professor
			\end{itemize}
			\item The admin validates and publishes the review.
		\end{tabenum} \\
		\textbf{Post-conditions} & The review is validated and visible to the public. \\
		\textbf{Alternative Steps} & \begin{tabitem}
			\item[2a.] The admin corrects a field by creating a new entry in the database.
			\item[2b.] If the fields cannot be linked or verified, the admin discards the review.
		\end{tabitem}
	\end{tabularx}
\end{usecasetable}

\begin{usecasetable}{8}{Validate Report}
	\begin{tabularx}{\textwidth}{l|X}
		\textbf{Level} & User Goal \\
		\textbf{Description} & The admin evaluates reported reviews and take moderation actions. \\
		\textbf{Actors} & Admin \\
		\textbf{Pre-conditions} & The admin is logged in. \\
		\textbf{Steps} &
		\begin{tabenum}
			\item The admin select the option to view pending reports.
			\item The admin select a report.
			\item The admin verifies if the review is to be removed.
			\item The review is deleted from the system.
		\end{tabenum} \\
		\textbf{Post-conditions} & The report has been processed and the moderation procedure is consistent in the database. \\
		\textbf{Alternative Steps} & \begin{tabitem}
			\item[3a.] If the review  doesn't contain any violation, the report is dismissed and the review remains visible.
			\item[4a.] \textbf{Extension (Ban User)}: If the content of the review meets the Ban Criteria the admin can map the review to the student ID and suspend his account???
		\end{tabitem}
	\end{tabularx}
\end{usecasetable}

\subsection{Mock-ups}

\begin{figure}[htbp]
	\centering
	\begin{subfigure}[t]{0.45\textwidth}
		\centering
		\setlength{\fboxsep}{0pt}\setlength{\fboxrule}{1pt}\color{lightgray}
		\fbox{\includegraphics[width=\textwidth]{./images/landing-page-mobile-crop.png}}
		\caption{Mock-up \#1: landing page. From this page the user is able to start browsing the review corpus and login.}
		\label{mock-up-1-landing-page}
	\end{subfigure}
	\hfill % Inserisce spazio elastico tra le due foto
	\begin{subfigure}[t]{0.45\textwidth}
		\centering
		\setlength{\fboxsep}{0pt}\setlength{\fboxrule}{1pt}\color{lightgray}
		\fbox{\includegraphics[width=\textwidth]{./images/browse-page-mobile-crop.png}}
		\caption{Mock-up \#2: browsing page. After clicking on the Browse button in the landing page, the user is brought to this page to explore the entire collection of published reviews. The user can also apply filters to narrow the search.}
		\label{mock-up-2-browsing-page}
	\end{subfigure}
	\vspace{1em}
	\caption{Guest's View of the platform.}
\end{figure}

\begin{figure}[htbp]
	\centering
	\begin{subfigure}[t]{0.45\textwidth}
		\centering
		\setlength{\fboxsep}{0pt}\setlength{\fboxrule}{1pt}\color{lightgray}
		\fbox{\includegraphics[width=\textwidth]{./images/auth-page-step-1-mobile.png}}
		\caption{Mock-up \#3: first step of the authentication procedure. The user needs to enter his personal institutional UniFi email.}
		\label{mock-up-3-auth-page-step-1}
	\end{subfigure}
	\hfill % Inserisce spazio elastico tra le due foto
	\begin{subfigure}[t]{0.45\textwidth}
		\centering
		\setlength{\fboxsep}{0pt}\setlength{\fboxrule}{1pt}\color{lightgray}
		\fbox{\includegraphics[width=\textwidth]{./images/auth-page-step-2-mobile.png}}
		\caption{Mock-up \#4: second step of the authentication procedure. A 6 digit code will be sent to the email address, then the user needs to fill the field with that code in order to complete the procedure and log int his account.}
		\label{mock-up-4-auth-page-step-2}
	\end{subfigure}
	\caption{Pages regarding the authentication procedure.}
\end{figure}

\begin{figure}[htbp]
	\centering
	\begin{subfigure}[t]{0.45\textwidth}
		\centering
		\setlength{\fboxsep}{0pt}\setlength{\fboxrule}{1pt}\color{lightgray}
		\fbox{\includegraphics[width=\textwidth]{./images/admin-control-panel-mobile-crop.png}}
		\caption{Mock-up \#5: Admin control panel landing page. From this page the admin can see all the pending matters and operate accordingly.}
		\label{mock-up-5-admin-control-panel}
	\end{subfigure}
	\hfill % Inserisce spazio elastico tra le due foto
	\begin{subfigure}[t]{0.45\textwidth}
		\centering
		\setlength{\fboxsep}{0pt}\setlength{\fboxrule}{1pt}\color{lightgray}
		\fbox{\includegraphics[width=\textwidth]{./images/admin-edit-page-mobile-crop.png}}
		\caption{Mock-up \#6: review validation edit page. From this form the admin is able to manage the verification of the drafted reviews and decide to publish them or delete them.}
		\label{mock-up-6-admin-edit-panel}
	\end{subfigure}
	\caption{Admin control panel.}
\end{figure}

\begin{figure}[htbp]
	\centering
	\begin{subfigure}[t]{0.45\textwidth}
		\centering
		\setlength{\fboxsep}{0pt}\setlength{\fboxrule}{1pt}\color{lightgray}
		\fbox{\includegraphics[width=\textwidth]{./images/review-form-page-mobile-crop-1.png}}
		\caption{Mock-up \#7: review form part I. The user needs to fill the form in order to try to publish a review.}
		\label{mock-up-7-review-form}
	\end{subfigure}
	\hfill % Inserisce spazio elastico tra le due foto
	\begin{subfigure}[t]{0.45\textwidth}
		\centering
		\setlength{\fboxsep}{0pt}\setlength{\fboxrule}{1pt}\color{lightgray}
		\fbox{\includegraphics[width=\textwidth]{./images/review-form-page-mobile-crop-2.png}}
		\caption{Mock-up \#8: review from part II.}
	\end{subfigure}
	\caption{Form for the review composition.}
\end{figure}

\subsection{Navigation Diagram}

\subsection{Class Diagram}

\subsection{Relational Model}

\begin{figure}[H]
	\centering
	\begin{tikzpicture}
		[every entity/.style={fill=blue!20,draw=blue,thick},
		every relationship/.style={fill=orange!20,draw=orange,thick,aspect=1.5},
		isa arrow/.style={thick,-{Triangle[open, length=8pt, width=8pt]}},
		attr/.style={circle, draw, fill=white, inner sep=0pt, minimum size=5pt, font=\footnotesize},
		pk/.style={circle, draw, fill=black, inner sep=0pt, minimum size=5pt, font=\footnotesize}]
		
		% User
		\node[entity](user){User};
		\node[pk, label=left:id](userId)[above left = 0.5cm of user]{} edge (user);
		\node[attr, label=left:isBanned](userIsBanned)[left = 0.5cm of user]{} edge (user);
		
		\node[relationship](files)[below = of user]{Files} edge node[right] {1} (user);
		\node[entity](report)[below = of files]{Report} edge node[right] {0..n} (files);
		\node[pk, label=above:id](reportId)[above left = 0.5cm of report]{} edge (report);
		\node[attr, label=above:date](reportDate)[left = 0.5 cm of report]{} edge (report);
		
		\node[relationship](writes)[right = of user]{Writes} edge node[above] {1} (user);
		
		% Relazione Up-voted posizionata tra Writes e About
		\node[relationship](upvoted)[below = 0.5cm of writes]{Up-voted};
		\draw[-] (user) -- node[below left] {1} (upvoted);
		
		\node[relationship](about)[right = of report]{About} edge node[above] {0..n} (report);
		
		\node[entity](review)[right = of about]{Review};
		\node[pk, label=right:id](reviewId)[above right = 0.75cm and 0.5cm of review]{} edge (review);
		\node[attr, label=right:date](reviewDate)[above right = 0cm and 0.75cm of review]{} edge (review);
		\node[attr, label=right:rating](reviewRating)[right = 0.5cm of review]{} edge (review);
		\node[attr, label=right:difficulty](reviewDifficulty)[below right = 0.75cm and 0.5cm of review]{} edge (review);
		\node[attr, label=right:comment](reviewComment)[below right = 0cm and 0.75cm of review]{} edge (review);
		
		\draw[-] (writes) -| node[below right] {0..n} (review);
		\draw[-] (about) -- node[above right] {1} (review);
		
		% Collegamento Up-voted a Review
		\draw[-] (upvoted) -- node[above] {0..1} (review);
		
		\node[entity](draftReview)[below = 2cm of review, xshift=-2cm]{Drafted Review};
		\draw[isa arrow] (draftReview) -- ++(0, 1) -| (review);
		\node[attr, label=left:school](draftSchool)[above left = 0.75cm and 0.5cm of draftReview]{} edge (draftReview);
		\node[attr, label=left:degree](draftDegree)[above left = 0cm and 0.75cm of draftReview]{} edge (draftReview);
		\node[attr, label=left:course](draftCourse)[below left = 0cm and 0.75cm of draftReview]{} edge (draftReview);
		\node[attr, label=left:professor](draftProfessor)[below left = 0.75cm and 0.5cm of draftReview]{} edge (draftReview);
		\node[entity](publishedReview)[below = 2cm of review, xshift=2cm]{Published Review};
		\draw[isa arrow] (publishedReview) -- ++(0, 1) -| (review);
		
		
		\node[entity](admin)[right = 7cm of user]{Admin};
		\node[pk, label=right:id](adminId)[above right = 0.5cm of admin]{} edge (admin);
		\node[attr, label=right:email](adminEmail)[right = 0.5cm of admin]{} edge (admin);
		
		% School, degress, course, professor
		\node[relationship](teaches)[below = 2cm of publishedReview]{Teaches} edge node[right] {0..n} (publishedReview);
		\node[entity](prof)[below = of teaches]{Professor} edge node[right] {1..n} (teaches);
		\node[pk, label=right:id](profId)[above right = 0.5cm of prof]{} edge (prof);
		\node[attr, label=right:name](profName)[right = 0.5cm of prof]{} edge (prof);
		\node[attr, label=right:surname](profSurname)[below right = 0.5cm of prof]{} edge (prof);
		\node[entity](course)[left = of teaches]{Course} edge node[above] {1..n} (teaches);
		\node[pk, label=below:id](courseId)[below left = 0.5cm of course]{} edge (course);
		\node[attr, label=below:name](courseName)[below right = 0.5cm of course]{} edge (course);
		\node[relationship](contains)[left = of course]{Contains} edge node[above] {1..n} (course);
		\node[entity](degree)[left = of contains]{Degree} edge node[above] {1} (contains);
		\node[pk, label=left:id](degreeId)[above left = 0.5cm of degree]{} edge (degree);
		\node[attr, label=left:name](degreeName)[left = 0.5cm of degree]{} edge (degree);
		\node[attr, label=left:type](degreeType)[below left = 0.5cm of degree]{} edge (degree);
		\node[relationship](offers)[above = of degree]{Offers} edge node[right] {1..n} (degree);
		\node[entity][above = of offers](school){School} edge node[right] {1} (offers);
		\node[pk, label=left:id](schoolId)[above left = 0.5cm of school]{} edge (school);
		\node[attr, label=left:name](schoolName)[left = 0.5cm of school]{} edge (school);
		
	\end{tikzpicture}
	\caption{E-R Model}
\end{figure}
